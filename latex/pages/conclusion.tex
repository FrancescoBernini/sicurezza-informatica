La sicurezza e l'affidabilità del software rappresentano sfide fondamentali nel panorama tecnologico moderno, specialmente per i sistemi safety-critical. L'analisi dei principali incidenti informatici ha dimostrato come errori di programmazione, spesso legati alla gestione della memoria e della concorrenza, possano avere conseguenze devastanti. Per mitigare questi rischi, sono state sviluppate diverse strategie, tra cui linee guida per linguaggi esistenti o modifiche safe degli stessi, analizzatori statici in grado di garantire proprietà di safety, nuovi paradigmi di programmazione come quello introdotto da Rust. Quest'ultimo, grazie al sistema di ownership, borrowing e lifetimes, fornisce garanzie di memory safety senza la necessità di un garbage collector, rendendolo una scelta promettente per lo sviluppo efficiente di software critico. Tuttavia, l'adozione di Rust e di altre metodologie sicure richiede un cambiamento di mentalità e un adeguato addestramento degli sviluppatori. In futuro, l'evoluzione di questi strumenti potrebbe ulteriormente migliorare la sicurezza del software, rendendo meno probabili gli errori umani e aumentando la resilienza dei sistemi informatici. L'integrazione di queste soluzioni nei processi di sviluppo rappresenta un passo essenziale per ridurre i rischi e garantire un futuro tecnologico più sicuro e affidabile.
