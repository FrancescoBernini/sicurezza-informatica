Nel contesto dello sviluppo software, safety e security sono due concetti strettamente correlati ma distinti.

La \textbf{safety} si riferisce alla capacità di un sistema di evitare guasti che possano portare a conseguenze pericolose, garantendo che il software si comporti in modo prevedibile anche in condizioni anomale.
Nei linguaggi di programmazione, la safety si traduce nell’adozione di meccanismi che riducono la probabilità di errori critici, come la \textit{type safety} (controllo rigoroso dei tipi di dato), l’assenza di \textit{undefined behavior} (\texttt{UB}) e l’uso di strumenti per la verifica formale del codice. Linguaggi come \texttt{Ada} e \texttt{Rust} offrono forti garanzie intrinseche in questo senso, con controlli statici avanzati e sistemi di gestione della memoria sicuri. Per quanto riguarda i linguaggi maggiormente utilizzati attualmente negli ambiti dove è richiesta anche un minimo di performance oltre alla safety, troviamo linguaggi come \texttt{C} e \texttt{C++} che non impongono il controllo rigoroso dei tipi e permettono operazioni non definite dal linguaggio (\texttt{UB}) aumentando il rischio di comportamenti imprevedibili. Tuttavia, essi offrono tutti i costrutti per applicare la safety, ma la responsabilità è affidata al programmatore.

La \textbf{security}, invece, riguarda la protezione del software da attacchi malevoli e accessi non autorizzati. Un sistema è \textit{secure} se resiste a tentativi di compromissione, proteggendo dati e funzionalità da utenti non autorizzati o codice dannoso.
Nei linguaggi di programmazione, la security è anch'essa influenzata dalla gestione della memoria (assenza di vulnerabilità come \textit{buffer overflow} o \textit{use-after-free}), dai meccanismi di isolamento tra processi (\textit{multithreading}) e, inoltre, dalla disponibilità di strumenti per la gestione sicura delle credenziali e della crittografia. Linguaggi moderni come \texttt{Rust} e \texttt{Go} sono progettati con un forte focus sulla sicurezza, prevenendo classi di vulnerabilità comuni.

Sebbene safety e security abbiano obiettivi diversi, esistono molte sovrapposizioni.
Ad esempio, una vulnerabilità di sicurezza può compromettere la safety di un sistema: un attacco informatico ad un software aeronautico scritto in \texttt{C}, che sfrutti un buffer overflow, potrebbe manipolare i dati di controllo con conseguenze catastrofiche. Per questo motivo, i linguaggi di programmazione utilizzati in contesti critici devono essere progettati per minimizzare sia i rischi legati alla safety che quelli legati alla security, attraverso controlli statici, verifiche formali e restrizioni sulle operazioni potenzialmente pericolose.