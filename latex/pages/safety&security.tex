Nel contesto dello sviluppo software, safety e security sono due concetti strettamente correlati ma distinti.

La safety si riferisce alla capacità di un sistema di evitare guasti che possano portare a conseguenze pericolose, garantendo che il software si comporti in modo prevedibile anche in condizioni anomale. Questo aspetto è cruciale nei sistemi safety-critical, come quelli avionici, automotive o biomedicali, dove un malfunzionamento potrebbe causare danni fisici o mettere a rischio vite umane.
Nei linguaggi di programmazione, la safety si traduce nell’adozione di meccanismi che riducono la probabilità di errori critici, come la type safety (controllo rigoroso dei tipi di dato), l’assenza di undefined behavior (UB) e l’uso di strumenti per la verifica formale del codice. Linguaggi come Ada e Rust offrono forti garanzie intrinseche in questo senso, con controlli statici avanzati e sistemi di gestione della memoria sicuri. Per quanto riguarda i linguaggi maggiormente utilizzati attualmente negli ambiti dove è richiesta anche un minimo di performance oltre alla safety, troviamo linguaggi come C e C++ che non impongono il controllo rigoroso dei tipi e permettono operazioni non definite dal linguaggio (UB) aumentando il rischio di comportamenti imprevedibili. Essi tuttavia, offrono tutti i costrutti per applicare la safety scaricando la responsabilità sul programmatore.

La security, invece, riguarda la protezione del software da attacchi malevoli e accessi non autorizzati. Un sistema è secure se resiste a tentativi di compromissione, proteggendo dati e funzionalità da utenti non autorizzati o codice dannoso.
Nei linguaggi di programmazione, la security è influenzata dalla gestione della memoria, dai meccanismi di isolamento tra processi (multithreading), dall’assenza di vulnerabilità come buffer overflow o use-after-free e dalla disponibilità di strumenti per la gestione sicura delle credenziali e della crittografia. Linguaggi moderni come Rust e Go sono progettati con un forte focus sulla sicurezza, prevenendo classi di vulnerabilità comuni. D'altra parte, linguaggi come C e C++ espongono gli sviluppatori a rischi maggiori poiché non forniscono protezioni integrate contro errori di gestione della memoria, lasciando anche in questo caso la responsabilità interamente al programmatore.

Sebbene safety e security abbiano obiettivi diversi, esistono molte sovrapposizioni.
Ad esempio, una vulnerabilità di sicurezza può compromettere la safety di un sistema: un attacco informatico su un software aeronautico scritto in C, che sfrutti un buffer overflow, potrebbe manipolare i dati di controllo con conseguenze catastrofiche. Per questo motivo, i linguaggi di programmazione utilizzati in contesti critici devono essere progettati per minimizzare sia i rischi legati alla safety che quelli legati alla security, attraverso controlli statici, verifiche formali e restrizioni sulle operazioni potenzialmente pericolose.