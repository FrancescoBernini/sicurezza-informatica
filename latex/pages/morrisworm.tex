Il Morris Worm, rilasciato nel 1988, è considerato il primo worm informatico su Internet.
Inizialmente concepito con l'idea di dimostrare l'esistenza di vulnerabilità nei sistemi Unix facilmente sfruttabili, il programma venne lanciato con la capacità di replicarsi su altre macchine attraverso la rete e di autodistruggersi allo spegnimento delle stesse. Tuttavia, all'epoca, lo spegnimento dei computer avveniva raramente e inoltre vi era un possibilità che esso si replicasse più volte sulla stessa macchina. Questi fattori portarono il programma ad avere un effetto devastante: infettò circa 6.000 computer (circa il 10\% dell'allora Internet) causando rallentamenti e crash di sistemi causando praticamente uno tra i primi "attacchi" DoS (Denial of Service).

Un worm è un tipo di malware che, a differenza di un virus che necessita dell'interazione dell'utente per essere eseguito, si replica e diffonde autonomamente da un computer all'altro sfruttando vulnerabilità di sicurezza come falle nei protocolli di rete o nei software di sistema senza bisogno di intervento umano.

Questo, in particolare, sfruttava una vulnerabilità di buffer overflow presente nel demone \texttt{finger}, un programma utilizzato per recuperare informazioni sugli utenti di un sistema Unix. Il codice, scritto in C, non eseguiva controlli adeguati sulla dimensione dei dati ricevuti, permettendo all'attaccante di scrivere dati oltre i limiti del buffer e sovrascrivere la memoria adiacente, potenzialmente eseguendo codice arbitrario. Il worm sfruttava anche altre falle di sicurezza, come una debolezza nel comando \texttt{sendmail}, che consentiva l'esecuzione remota di codice non autorizzato.

Il Morris Worm portò alla creazione del CERT (Computer Emergency Response Team) e alla consapevolezza della necessità di migliori pratiche di sicurezza nella programmazione. Questo evento evidenziò l'urgenza di protezioni contro attacchi simili e stimolò lo sviluppo di mitigazioni come ASLR (Address Space Layout Randomization), che rende più difficile prevedere la posizione della memoria utilizzata da un programma, e Stack Canaries, una tecnica che protegge la memoria dello stack da sovrascritture accidentali o malevole.