Negli ultimi anni, la sicurezza e l'affidabilità del software sono diventate questioni centrali nello sviluppo di sistemi informatici, specialmente in contesti safety-critical come l'aerospaziale, l'automotive e il biomedicale. La maggior parte delle vulnerabilità software deriva da errori nella gestione della memoria e della concorrenza, con conseguenze potenzialmente catastrofiche. Questo lavoro esplora le differenze tra safety e security del codice, evidenziando i principali problemi legati alla programmazione in linguaggi tradizionali come C e C++. Vengono analizzati casi storici di fallimenti software, come il Morris Worm, il disastro del razzo Ariane 5 e gli incidenti legati al Therac-25, per illustrare le implicazioni di un codice insicuro. Successivamente, si esaminano le soluzioni adottate nel tempo, tra cui tecniche di analisi statica, sottoinsiemi sicuri di linguaggi esistenti (come MISRA C e Safe C++) e l'approccio innovativo del linguaggio Rust, che offre meccanismi integrati per la gestione sicura della memoria senza il costo di un garbage collector. L'obiettivo di questa ricerca è dimostrare l'importanza del codice sicuro in ambiti safety-critical e di informare su alcune delle tecniche di mitigazione esistenti allo stato dell'arte.
