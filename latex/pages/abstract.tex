La sicurezza e l'affidabilità del software sono questioni centrali nello sviluppo di sistemi informatici, specialmente in contesti \textit{safety-critical} come l'aerospaziale, l'automotive e il biomedicale.
La maggior parte delle vulnerabilità software derivano da errori nella gestione della memoria e della concorrenza, con conseguenze potenzialmente catastrofiche. Questo lavoro esplora le differenze tra \textbf{safety} e \textbf{security} del codice, evidenziando i principali problemi legati alla programmazione in linguaggi tradizionali come \texttt{C} e \texttt{C++}. Vengono analizzati casi storici di fallimenti software, come il Morris Worm, il disastro del razzo Ariane 5 e gli incidenti legati al Therac-25, per illustrare le implicazioni di un codice insicuro. Successivamente, si esaminano le soluzioni adottate nel tempo, tra cui tecniche di analisi statica, definizione di sottoinsiemi sicuri di linguaggi esistenti (come \texttt{MISRA C} e \texttt{Safe C++}) e l'approccio innovativo del linguaggio \texttt{Rust}, che offre meccanismi integrati per la gestione sicura della memoria senza il costo di un \textit{garbage collector}. L'obiettivo di questa ricerca è quello di evidenziare l'importanza del codice sicuro e di informare su alcune delle tecniche di mitigazione del rischio esistenti allo stato dell'arte.
