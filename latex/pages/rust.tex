Rust è un linguaggio di programmazione moderno progettato per garantire sicurezza e prestazioni senza l'uso di un garbage collector. La gestione della memoria in Rust si basa su tre concetti fondamentali: \textbf{ownership}, \textbf{borrowing} e \textbf{lifetimes}. Questi meccanismi permettono di prevenire errori comuni come use-after-free, memory leaks e data races in ambienti concorrenti.

\subsubsection{Ownership}

L'\textbf{ownership} è il principio cardine della gestione della memoria in Rust. Ogni valore in Rust ha un proprietario unico e quando il proprietario esce dallo scope, la memoria viene automaticamente deallocata.

\begin{algorithm}[ht]
    \caption{Esempio dimostrativo dell'ownership}
    \label{lst:ownership}
    \begin{lstlisting}[language=Rust, style=colouredRust]
fn main() {
    let s1 = String::from("Ciao");
    let s2 = s1; // proprieta' trasferita a s2
   
    // println!("{}", s1);  // Errore!
    println!("{}", s2);  // Stampa "Ciao"
}
\end{lstlisting}
\end{algorithm}


\subsubsection{Borrowing}

Poiché l'ownership di un valore non può essere trasferita automaticamente, Rust consente di prendere in prestito (\textbf{borrowing}) i valori attraverso le references (\texttt{\&}) consentendo di accedere ai dati senza cambiarne il proprietario.

Rust distingue tra borrowing immutabile e mutabile: non si possono avere più riferimenti mutabili simultanei e non ci possono essere riferimenti mutabili se ci sono già riferimenti immutabili a una stessa variabile.

\begin{algorithm}[ht]
    \caption{Esempio di borrowing immutabile}
    \label{lst:borrowing_immutabile}
    \begin{lstlisting}[language=Rust, style=colouredRust]
fn main() {
    let s1 = String::from("Ciao");
    let len = calcola_lunghezza(&s1);  // Presta s1 in modo immutabile

    println!("La lunghezza di '{}' e' {}.", s1, len);
}
        
fn calcola_lunghezza(s: &String) -> usize {
    s.len()
}
\end{lstlisting}
\end{algorithm}

In questo esempio, s1 viene prestata in modo immutabile alla funzione calcola\_lunghezza. Dopo la chiamata, s1 rimane valida e può essere utilizzata.

\begin{algorithm}[ht]
    \caption{Esempio di borrowing mutabile}
    \label{lst:borrowing_mutabile}
    \begin{lstlisting}[language=Rust, style=colouredRust]
fn main() {
    let mut s1 = String::from("Ciao");
    modifica_stringa(&mut s1);  // Presta s1 in modo mutabile
        
    println!("{}", s1);  // Stampa "Ciao, Mondo!"
}
        
fn modifica_stringa(s: &mut String) {
    s.push_str(", Mondo!");
}
\end{lstlisting}
\end{algorithm}

In questo esempio, s1 viene prestata in modo mutabile alla funzione modifica\_stringa, che modifica la stringa aggiungendo ``, Mondo!''.

\begin{algorithm}[ht]
    \caption{Esempio di borrowing mutabile quando già immutabile}
    \label{lst:borrowing_mutabile_e_immutabile}
    \begin{lstlisting}[language=Rust, style=colouredRust]
fn main() {
    let mut s = String::from("Ciao");
        
    let r1 = &s;  // Prestito immutabile
    let r2 = &s;  // Prestito immutabile
    // let r3 = &mut s;  // Questo causerebbe un errore di compilazione
        
    println!("{}, {}", r1, r2);
}
\end{lstlisting}
\end{algorithm}


\subsubsection{Lifetimes}

I lifetimes garantiscono a compile-time che le references siano sempre valide e non puntino a memoria deallocata.

\begin{algorithm}[ht]
    \caption{Esempio gestione lifetime}
    \label{lst:lifetimes}
\begin{lstlisting}[language=Rust, style=colouredRust]
fn main() {
    let string1 = String::from("lunga stringa");
    let string2 = "xyz";
    
    let risultato = longest(string1.as_str(), string2);
    println!("La stringa piu' lunga e' {}", risultato);
}
    
fn longest<'a>(x: &'a str, y: &'a str) -> &'a str {
    if x.len() > y.len() {
        x
    } else {
        y
    }
}
\end{lstlisting}
\end{algorithm}

In questo esempio, la funzione longest restituisce il riferimento alla stringa più lunga. Il ciclo di vita 'a garantisce che i riferimenti x e y siano validi per lo stesso periodo.

Questi concetti rendono Rust particolarmente adatto per lo sviluppo di sistemi sicuri e applicazioni ad alte prestazioni secondo le guidelines pubblicate già nel 2023 dalla CISA (Cybersecurity \& Infrastructure Security Agency), agenzia di rilevanza internazionale facente parte del dipartimento di Sicurezza Interna degli Stati Uniti \cite{CISA}.
